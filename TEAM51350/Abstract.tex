\begin{abstract}
The chief purpose of this study is to construct a model about the water consumption of a given area. In convention, water usage is divided into 3 parts: Industrial, Agricultural and Domestic. By considering each part of water usage as quantity which always adapt to the consumption of other human activities, We choose many fundamental quantities (such as population, PCGDP ...) to be fitted by proper functions as candidates to determine  each water consumption. Then specific correlation coefficients are calculated to verify the relations, in the process of which some irrelevant quantities are eliminated. By fitting the remaining quantities with each part of water usage, we can calculate the water usage per capita to predict the water usage situation in the following years.

After that, specific intervention plans will be designed under the simulation results to improve the future water supply ability. The simulation can then be conducted again under the new situations.

We put China as the main example of our analysis and come to the conclusion that China is in and will still be in water scarcity without intervention plan, and if proper intervention can be done, the situation will become less stress.

\begin{keywords}
    Water consumption, Water scarcity, Water policy, Data mining.
\end{keywords}
\end{abstract}
